
\documentclass[a4paper,11pt]{article}

%\documentclass[12pt]{article}

% Math Related Packages
\usepackage{amsmath}
\usepackage{amssymb}
%\usepackage{bbm}
%\usepackage{cancel}
%\usepackage{mathtools}
\usepackage{slashed}
\usepackage[numbers,sort&compress]{natbib}
%%%%%%%%%%%%%%%%%%%%%%%%%%%%%%%%%%%%%%%%%%%%%%%%%%%%%%%%
\usepackage{hyperref}
%%%%
\usepackage{cleveref}
\crefname{equation}{Eq.}{Eqs.}
\crefname{figure}{Fig.}{Figs.}
\crefname{table}{Table}{Tables}
\crefname{section}{Section}{Sections}
\newcommand{\crefrangeconjunction}{--}
%%%


%%%%%%%%%%%%%%%%%%%%%%%%%%%%%%%%%%%%%%%%%%%%%%%%%%%%%%%
% Formatting Related Packages
\usepackage[letterpaper,margin=1in,bottom=1in]{geometry}
\usepackage{float} % allows forcing plot and table locations with [H]
\usepackage{parskip} % skip spaces between paragraphs
\usepackage{tabulary} % table environment with auto word wrapping
\usepackage{color} % color for tracking edits
\usepackage{soul} % strikeout for tracking edits
\usepackage{subfigure}
\usepackage{graphicx}
\usepackage[section]{placeins} % keep figures inside their sections
\def\lhc2{LHC~Run~II}
\def\nsu{nonuniversal~supergravity~models~}
\def\dd{$\cdot\cdot$}
% Reference Package
\usepackage{cleveref}
% Macros
\newcommand{\code}[1]{\texttt{#1}}
\newcommand{\GeV}{~\textrm{GeV}}
%\bibliographystyle{suj_long}
\bibliographystyle{suj}
\newcommand{\pb}      {~\mathrm{pb}}
\def\nj{$n_{jet}$}
\def\njs{$n_{jet}^*$}
\def\co{coannihilation~}
\def\pts{$P_T^{\rm * miss}$}
\def\pt{\not\!\!{P_T}}
\def\no{\nonumber\\}
\def \chan{\widetilde{\chi}}
\def \cha{\widetilde{\chi}^{\pm}_1}
\def \chb{\widetilde{\chi}^{\pm}_2}
\def \na{\widetilde{\chi}^{0}}
\def \nb{\widetilde{\chi}^{0}_2}
\def \nc{\widetilde{\chi}^{0}_3}
\def \nd{\widetilde{\chi}^{0}_4}
\def \g{\tilde{g}}
\def \ql{\widetilde{q}_L}
\def \qr{\widetilde{q}_R}
\def \dl{\widetilde{d}_L}
\def \dr{\widetilde{d}_R}
\def \ul{\widetilde{u}_L}
\def \ur{\widetilde{u}_R}
\def \ccl{\widetilde{c}_L}
\def \ccr{\widetilde{c}_R}
\def \ssl{\widetilde{s}_L}
\def \ssr{\widetilde{s}_R}
\def \ta{\widetilde{t}_1}
\def \tb{\widetilde{t}_2}
\def \ba{\widetilde{b}_1}
\def \bb{\widetilde{b}_2}
\def \sta{\widetilde{\tau}_1}
\def \stb{\widetilde{\tau}_2}
\def \smr{\widetilde{\mu}_R}
\def \ser{\widetilde{e}_R}
\def \sml{\widetilde{\mu}_L}
\def \sel{\widetilde{e}_L}
\def \slr{\widetilde{l}_R}
\def \sll{\widetilde{l}_L}
\def \snl{\widetilde{\nu}_{\tau}}
\def \snm{\widetilde{\nu}_{\mu}}
\def \sne{\widetilde{\nu}_{e}}
\def \hc{H^{\pm}}
\def \lra{\longrightarrow}
\def \ETmiss{${\not\!\!{E_T}}~$}
\def \missET{${\not\!\!{E_T}}$}
\def \mh{m_{1/2}}
\def\.4{\vspace{-.5cm}}
\newcommand{\ifb}{~\textrm{fb}^{-1}}


\def\beq{\begin{equation}}
\def\be{\begin{equation}}
\def\beqn{\begin{eqnarray}}
\def\ee{\end{equation}}
\def\eeq{\end{equation}}
\def\eeqn{\end{eqnarray}}
\newcommand{\bl}[1]{{\color{blue}{#1}}}
\newcommand{\pn}[1]{{\color{red}{#1}}}
\newcommand{\gr}[1]{{\color{green}{#1}}}
\def\co{coannihilation~}
\def\TeV{\rm{TeV}}

\def\co{coannihilation~}
\def\G{\tilde G}
\def\mP{M_{\rm Pl}}

\begin{document}


%\documentclass[10pt]{article}
%\usepackage{geometry}                % See geometry.pdf to learn the layout options. There are lots.
%\geometry{letterpaper}                   % ... or a4paper or a5paper or ... 
%%\geometry{landscape}                % Activate for for rotated page geometry
%%\usepackage[parfill]{parskip}    % Activate to begin paragraphs with an empty line rather than an indent
%\usepackage{graphicx}
%\usepackage{amssymb}
%\usepackage{epstopdf}
%\DeclareGraphicsRule{.tif}{png}{.png}{`convert #1 `dirname #1`/`basename #1 .tif`.png}
%\def\bv{\begin{verbatim}}
%\def\ev{\end{verbatim}}
%\newcommand{\pn}[1]{{\color{blue}{#1}}}
%%%%%%%%%%%%%%%%%%%%%%%%%%%%%%%%%%%
%\newcommand{\pr}[1]{{\color{red}{#1}}}
%%%%%%%%%%%%%%%%%%%%%%%%%%%%%%%%%%%%
%\def\vs{\vspace{0.22cm}}
%\title{\small  Reply to the second report of referee B}
%%\author{}
%\date{}                                           % Activate to display a given date or no date
%\vspace{-5cm}
%\begin{document}
%\vspace{-3cm}
%\maketitle

 \begin{center}
 Reply to report of referee
  \end{center}

{\it We thank the referee for his/her comments and suggestions. 
Below are our brief replies to the referee report.}


\noindent
Referee:\\
 The paper studies inflation and non-Gaussianity in a two-field supersymmetric DBI model. The paper is of some interest, but I have several questions and a few suggestions for improvement before I can formulate a recommendation.\\
 
1) What value of m in (3.11) is taken in the concrete numerical evaluations?\\
 \noindent {\it Reply\\
 \pn{
 In the analysis we have taken $m=6$. }
 }
 
 
 
2) Nearly all relations in the paper are given in an implicit way, (for example keeping derivatives of the superpotential unspecified). In my opinion, the paper would improve if it showed the following explicit results:\\
 
 {\it Reply\\
 \pn{The derivatives of the superpotential first appear in Eq. (3.12) because their explicit forms were not needed in the
 immediate discussion that follows Eq. (3.12). They were made explicit in Eqs.(3.20), (3.21), (3.23)-(3.26). 
 However, we are happy to make them explicit immediately after Eq. (3.12) which we have done in the revised version.}
 
 }
 
 
 
 
 
2.1) A check that the real directions $\rho_1$  and  $\rho_2$ are indeed stable, also in presence of $W_{sb}$ and of subleading terms in 
 $1/T$. To ensure that these direction are stable during inflation, and to avoid potential problems with isocurvature modes from inflation, one should verify that the real directions are sufficiently heavy, namely parametrically heavier than the Hubble rate. One should provide the explicit expressions for the masses (from the curvature of the scalar potential in the minimum) in terms of the parameters of the model, and verify that they are greater than H (also given explicitly in terms of the parameters of the model).\\
 \noindent {\it Reply\\
 \pn{
 The stability conditions for the real directions $\rho_1$ and $\rho_2$ are given by Eq. (3.18) of the paper. This condition satisfies 
 stability condition to all orders in the $1/T$ expansion as can be seen in Eq. (3.17). \pn{We have also checked numerically  that 
 this condition is  the stability condition for the full potential given by Eqs. (10.4)-(10.9) even without the expansion in $1/T$. }
 Returning to Eq. (3.18), satisfaction of this condition requires satisfaction of Eqs. (3.20) and (3.21).  In general for arbitrary $m$,
Eqs. (3.20) and (3.21) cannot be solved analytically. However, we note the stability conditions contain two parts, one coming from
$W_s$ and the other $W_{sb}$. In the analysis we assume that $W_s$ is invariant under the $U(1)$ global symmetry,  and 
the global symmetry is broken by instanton or gaugino condensation effects by the term $W_{sb}$ and further that
$|W_{sb}| << W_{s}$. Thus in the absence of $W_{sb}$, $f_i$ is determined by $W_{s}$ alone and we arrange it have a 
mass the size the GUT scale. The effects of inclusion of $W_{sb}$ can be computed numerically or analytically in a perturbation 
expansion since the contribution of  $W_{sb}$ is much smaller than of $W_{s}$.  Thus the stability of the $\rho_1$ and $\rho_2$ 
directions in maintained when $W_{sb}$ is included which governs inflation. We have added a paragraph on it following Eq. (3.21). 
}}

 
 
2.2) A similar analysis should be done for the combination  $a_+$. After eq. (3.26), the paper states �$a_+$ undergoes fast roll.... thus we suppress $a_+$�. This point should be clarified. My understanding is that this sentence means that $a_+$ has a much larger mass than a?, so we can consistently set $a_+$ = 0. This should be shown explicitly, by computing explicitly the mass of this field (in terms of the parameters in the model) and showing that it is indeed greater than H.\\

\noindent {\it Reply\\
\pn{ This is one of the central points of the analysis, i.e., decomposition of the potential into slow roll vs fast roll where the slow roll potential
 depends  only on $a_-$ and the remaining part of the potential is controlled by $a_+$ and is fast roll. The reason we did not elaborate
 too much on  it in this paper is because this topic was  dealt with in great detail in our previous paper  ``Evidence for Inflation in an Axion Landscape,''  JHEP {\bf 1803}, 121 (2018). In this work it was shown that $a_+$ has a much larger mass than $a_-$ and further
 that the $a_+$ rolls an order of magnitude or more  faster than $a_-$ as shown by blue lines (fast roll) vs red lines (slow roll) in Fig.1
 and Fig.2 of JHEP {\bf 1803}, 121 (2018). However, we have added an additional text in the revised version of the paper 
 to explain this further.\\
  \pn{Max: May be we add  a figure showing fast roll and slow roll in the revised version.}
}}


2.3) It would be useful to see the explicit form of $V (a_-)$, when the other fields are set to the minimum. In this way the reader can see the explicit form of the inflaton potential used in the numerical evolutions. If many terms are present, it would be useful to single out the dominant ones (I would expect the usual cosine potential, plus corrections; please provide the expression for the dominant term and the leading correction).
2.4) The parametrization (3.19) suggests that $\rho_k = a_k = 0$ in the minimum of the potential. Therefore, as it is standard, the axion scale 
$f_k$ is obtained from the minimization of the scalar potential. Please provide the explicit relation that gives $f_k$ in terms of the parameters of the model.\\
\noindent {\it Reply\\ 
\pn{
This is an interesting question.  Indeed for conventional inflation models an explicit form of the potential plays a central role in the 
analysis. Unfortunately the potential by itself is not a useful quantity in the analysis for the Dirac-Born-Infeld case. Here the entire
Lagrangian enters in the analysis and the potential in isolation plays no role in the analysis. This can be seen from the Friedman equations
Eqs.(4.9) and (4.10) where the full Lagrangian and not the potential enters. Only in the case when we assume canonical kinetic energy and
no dependence of the potential on the derivatives of the fields that one recovers the conventional Friedman equations where 
the potential appears as shown in Eqs. (4.12).  However, we thank the referee for bringing this point up and we have added 
a few lines to explain this point after Eq. (4.12). Regarding $f_k$ a closed form solution for $f_k$ cannot be exhibited since it satisfies
a polynomial constraint of high order. However,  we have exhibited the explicit form of 
$f$ on the input parameters in Eq. (3.22) in a perturbation expansion on the symmetry breaking part.  }

}


3) There are different shapes of non-gaussianity (the equilateral and the local shapes being the most studied ones), and the limits on non-gaussianity strongly depend on the shape. By shape, we mean the dependence of the bispectrum on the ratios $k_2/k_1$ and 
$k_3/k_1$. The paper cites [64,65,66]. Eq. 11 of [64], which is also eq. 1 of [65], and eq. 17 of [66] defines the so called local shape. This is different from the shape of non-Gaussianity obtained in DBI models, which is very close to the equilateral shape (see for instance Section 2.1 of the Planck paper 1502.01592). Moreover the works [64,65,66], although very important, are rather outdated by now. So, this part of the paper should be updated, and the value of $f_{NL}$ should be compared to the current limits (see 1502.01592) or future forecasts on equilateral non-Gaussianity, not the local one.

\noindent {\it Reply\\ 
\pn{We agree that our references to the analysis of non-Gaussianity from the Planck experiment needs updating. We have
done so in the revised version.  The changes made appear after Eq.(5.12) and at the end of section 6. We have included
the reference to 1502.01592 and specified the current limits on $f_{NL}^{equil}$.
}}


4) I don�t understand eq.(6.1). The conventional way to find the end
of inflation is to find the moment at which $\ddot a = 0$ (second derivative of the
scale factor with respect to time). The number of e-folds during inflation is
typically defined from $a(N ) \equiv a_{end} e^{-N}$ , which gives  $\frac{dN}{dt} = -H$ (where $a_{end}$ 
is the value of the scale factor at the end of inflation, and H is the Hubble
rate). In this way, $N = 0$ at the end of inflation, while N is positive at
early times during inflation. How is eq. (6.1) used to determine the end of
inflation ?  What is the purpose of setting $\theta_n = \frac{1}{16}$?  Why don't the authors 
find from their numerical evolutions the point at which $\ddot a = 0$ ?\\
\noindent {\it Reply\\ 
\pn{ This is for Max to write.}
}



5) The scan shown in Figure 1 contains no real information, unless one
specifies what is varied among the different points. The only information I
could find on this regard is �we allow $\alpha_1$ to vary� at the end of page 12.
How is it varied ? (in which range ? Is it a uniform sampling in linear units
? In log units ?) What is chosen for the other parameters of the model
? (?,?,A,T). How is the correct normalization of the scalar perturbations
enforced? (this is typically done by fixing the overall scale of the scalar
potential). Are subleading terms in $\frac{1}{T}$  important ? The scan strategy should T
be completely described, so that any reader, if interested, could precisely reproduce those figures.\\

\noindent {\it Reply\\
\pn{This is for Max to write.
}}


6) The paper motivates the model from the trans-Planckian problem of axion inflation (the fact that we must take the axion scale 
$f > M_p$ in the original potential of natural inflation). It would be very interesting to provide also a scan of the values of f obtained for the various realizations presented in figures 1 and 2. Does this scan extend in the sub-Planckian region ? (the scan should be easily done, once the above questions 2.4 and 5 are answered).\\

\noindent {\it Reply\\
\pn{We agree with the suggestion of the referee. We have added a new figure 3 where we have presented 
 a scatter plot of $f_{eff}$ vs $f$ for the data sets of figures 1 and 2.\\~\\
  This for Max to write. 
  }





%
%\vs
%%\noindent
%Referee:

%

%\vs
%{\it Reply:  We agree with what the referee says.  }  \\

% \vs
%%\noindent
%Referee:     
%\\

%\vs
%{\it Reply:  
%}  \\

%\vs
%Referee:

%\\

%
%\vs
%{\it Reply:   

%} \\

%{\it 

%

%}
% 
%\vs
%%\noindent
%Referee:

%\\

%
%\vs
%{\it Reply: 

%   }  \\
%   
%\vs
%Referee:  

%
%\vs
%{\it Reply:   

%  }  \\
%  
%\vs
%%\noindent
%Referee:
%I

%
%\vs
%{\it Reply:   

%  }  \\
%%%%%%%%%%%%%%%%%%%%%%%%%%%%%%%%%%%%%%%%%%%%%%%%%%%%%%%%%%%%%%%
 
 
\end{document}  

